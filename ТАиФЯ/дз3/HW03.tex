\documentclass[12pt,a4paper]{article}%
\usepackage{amsthm}
\usepackage{amsmath}%
\usepackage{amsfonts}%
\usepackage{amssymb}%
\usepackage{graphicx}
\usepackage[T2A]{fontenc}
\usepackage[utf8]{inputenc}
\usepackage[english,russian]{babel}
\usepackage[linguistics]{forest}
%-------------------------------------------
\setlength{\textwidth}{7.0in}
\setlength{\oddsidemargin}{-0.35in}
\setlength{\topmargin}{-0.5in}
\setlength{\textheight}{9.0in}
\setlength{\parindent}{0.3in}

\newtheorem{theorem}{Theorem}
\newtheorem{task}[theorem]{Задача}
\addto\captionsrussian{\renewcommand*{\proofname}{Решение}}


\newcommand{\abovemath}[2]{\ensuremath{\stackrel{\text{#1}}{#2}}}
\newcommand{\aboveeq}[1]{\abovemath{#1}{=}}
\newcommand\bydef{\aboveeq{def}}
\begin{document}


\begin{flushright}
\textbf{Мирончик Павел, 8382 \\
\today}
\end{flushright}

\begin{center}
\textbf{Формальные языки\\
HW03 \\
Дедлайн: 23:59 15 ноября 2021} \\
\end{center}

\task{
    Привести однозначную контекстно-свободную грамматику для языка арифметических выражений над положительными целыми \emph{числами} с операциями \verb!+!, \verb!-!, \verb!*!, \verb!/!, \verb!^!, \verb!==!,\verb!<>!, \verb!<!, \verb!<=!, \verb!>!, \verb!>=! со следующими приоритетами и ассоциативностью:

  \begin{center}
    \begin{tabular}{ c | c }
      Наибольший приоритет & Ассоциативность  \\ \hline \hline
      \verb!^! & Правоассоциативна \\
     \verb!*!,\verb!/! & Левоассоциативна \\
     \verb!+!,\verb!-! & Левоассоциативна \\
     \verb!==!,\verb!<>!, \verb!<!,\verb!<=!, \verb!>!,\verb!>=! & Неассоциативна \\ \hline \hline
     Наименьший приоритет & Ассоциативность
    \end{tabular}
    \end{center}


    Неассоциативные операции встречаются только один раз: \verb!1 == 2! -- корректная строка, \verb!1 == 2 == 3!, \verb!(1 == 2) == 3!, \verb!1 < 2 > 3! --- некорректные строки
  }

\begin{proof}
  Отметим, что логические операции сравнения встречаются только один раз. Это означает, что они должны присутствовать 
  в единственном экземпляре, и слова без них считаются невалидными.

  \begin{center}
    \begin{align*}
      S &\to ABA \\
      B &\to \text{==} \mid \text{<>} \mid \text{<} \mid \text{<=} \mid \text{>} \mid \text{>=} \\
      A &\to (A) \mid AOA \mid N \\
      O &\to + \mid - \mid * \mid / \mid {^\land} \\
      N &\to 0 \mid 1 N_f \mid 2 N_f \mid 3 N_f \mid 4 N_f \mid 5 N_f \mid 6 N_f \mid 7 N_f \mid 8 N_f \mid 9 N_f \\
      N_f &\to N \mid 0 N_f \mid \varepsilon \\
    \end{align*}
  \end{center}
\end{proof}

\task {Привести грамматику из 1 задания в нормальную форму Хомского.}

\begin{proof}
  \leavevmode 
  \begin{center}
    \begin{align*}
      V_N &= S, A, B, O, N_f, D_*, C_*\\
      V_T &= 0 \ldots 9, {=}, {<}, {>}, +, -, *, /, {^\land}, {(}, {)}\\
      S &= S\\
      \\
      S &\to A D_1\\
      D_1 &\to B A\\
      B &\to C_{=} C_{=} \mid C_{<} C_{>} \mid {<} \mid C_{<} C_{=} \mid {>} \mid C_{>} C_{=}\\
      A &\to C_{(} D_2 \mid A D_3\\
      D_2 &\to A C_)\\
      D_3 &\to O A\\
      O &\to {+} \mid {-} \mid {*} \mid {/} \mid {^\land}\\
      A &\to 0 \mid C_1 N_f \mid C_2 N_f \mid \cdots \mid C_9 N_f\\
      N_f &\to C_0 N_f \mid C_1 N_f \mid \cdots \mid C_9 N_f \mid \varepsilon\\
      \forall x: C_x &\to x\\
    \end{align*}
  \end{center}
\end{proof}

\task {Промоделировать работу алгоритма CYK на грамматике из 2 задания на трех корректных строках не короче 7 символов и на трех некорректных строках. (Привести таблицы и деревья вывода)}

\begin{center}

\begin{center}
  
  \textbf{1. $1+12>3{^\land}10$ - входит} \bigskip

  \begin{tabular}{c|c|c|c|c|c|c|c|c}
    1&2 & 3 & 4 & 5 & 6 & 7 & 8 & 9 \\ \hline
    1&+&1&2&>&3&$^\land$&1&0 
  \end{tabular}\bigskip

  \begin{tabular}{c || c | c | c | c | c | c | c | c | c | }
      &1    &2    &3    &4    &5    &6    &7    &8    &9    \\  \hline \hline
    1 &$C_1|A$&$\varnothing$&$A$&$A$&$\varnothing$&$S$&$\varnothing$&$S$&$S$\\  \hline
    2 &&$O$&$D_3$&$D_3$&$\varnothing$&$\varnothing$&$\varnothing$&$\varnothing$&$\varnothing$\\  \hline
    3 &&&$C_1|A$&$A|N_f$&$\varnothing$&$S$&$\varnothing$&$S$&$S$\\  \hline
    4 &&&&$C_2|A$&$\varnothing$&$S$&$\varnothing$&$S$&$S$\\  \hline
    5 &&&&&$C_>|B$&$D_1$&$\varnothing$&$D_1$&$D_1$\\  \hline
    6 &&&&&&$C_3|A$&$\varnothing$&$A$&$A$\\  \hline
    7 &&&&&&&$O$&$D_3$&$D_3$\\  \hline
    8 &&&&&&&&$C_1|A$&$A$\\  \hline
    9 &&&&&&&&&$C_0|A$\\  \hline
  \end{tabular}
\end{center}

\bigskip

\begin{center}
  \begin{forest}
    [S
      [A
        [A
          [$C_1$ [1]]
          [$N_f$ [$\varnothing$]]
        ]
        [$D_3$
          [O [+]]
          [A 
            [$C_1$ [1]]
            [$N_f$ 
              [$C_2$ [2]]
              [$N_f$ [$\varnothing$]]
            ]
          ]
        ]
      ]
      [$D_1$
        [$B$ [>]]
        [A
          [A
            [$C_3$ [3]]
            [$N_f$ [$\varnothing$]]
          ]
          [$D_3$
            [O [$^\land$]]
            [A
              [$C_1$ [1]]
              [$N_f$ [$C_0$ [0]] [$N_f$ [$\varnothing$]]]
            ]
          ]
        ]
      ]
    ]
  \end{forest}
\end{center}
\bigskip
  
\textbf{2. $(1) == 0 * 0$ - входит} \bigskip

\begin{tabular}{c|c|c|c|c|c|c|c|c}
  1&2 & 3 & 4 & 5 & 6 & 7 & 8 \\ \hline
  (&1&)&=&=&0&*&0 
\end{tabular}\bigskip

\begin{tabular}{c || c | c | c | c | c | c | c | c | }
    &1    &2    &3    &4    &5    &6    &7    &8    \\  \hline \hline
  1 &$C_($&$\varnothing$&$A$&$\varnothing$&$\varnothing$&$S$&$\varnothing$&$S$ \\ \hline
  2 &&$C_1|N_f|A$&$D_2$&$\varnothing$&$\varnothing$&$\varnothing$&$\varnothing$&$\varnothing$ \\ \hline
  3 &&&$C_)$&$\varnothing$&$\varnothing$&$\varnothing$&$\varnothing$&$\varnothing$ \\ \hline
  4 &&&&$C_=$&$B$&$D_1$&$\varnothing$&$D_1$ \\ \hline
  5 &&&&&$C_=$&$\varnothing$&$\varnothing$&$\varnothing$ \\ \hline
  6 &&&&&&$C_0|N_f|A$&$\varnothing$&$A$ \\ \hline
  7 &&&&&&&$O$&$D_3$ \\ \hline
  8 &&&&&&&&$C_0|N_f|A$\\ \hline
\end{tabular}
\end{center}
\bigskip
  
...аналогичное дерево для этого примера, аналогичные таблицы для неверных примеров вроде $"1+2*3^\land4"$, за исключением 
того, что верхний левый элемент таблицы мы не сможем вывести из какой-либо пары предыдущих, поэтому слово не подойдет языку.


\end{document}
